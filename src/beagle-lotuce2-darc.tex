%  article.tex (Version 3.3, released 19 January 2008)
%  Article to demonstrate format for SPIE Proceedings
%  Special instructions are included in this file after the
%  symbol %>>>>
%  Numerous commands are commented out, but included to show how
%  to effect various options, e.g., to print page numbers, etc.
%  This LaTeX source file is composed for LaTeX2e.

%  The following commands have been added in the SPIE class 
%  file (spie.cls) and will not be understood in other classes:
%  \supit{}, \authorinfo{}, \skiplinehalf, \keywords{}
%  The bibliography style file is called spiebib.bst, 
%  which replaces the standard style unstr.bst.  

\documentclass[]{spie}  %>>> use for US letter paper
%%\documentclass[a4paper]{spie}  %>>> use this instead for A4 paper
%%\documentclass[nocompress]{spie}  %>>> to avoid compression of citations
%% \addtolength{\voffset}{9mm}   %>>> moves text field down
%% \renewcommand{\baselinestretch}{1.65}   %>>> 1.65 for double spacing, 1.25 for 1.5 spacing 
%  The following command loads a graphics package to include images 
%  in the document. It may be necessary to specify a DVI driver option,
%  e.g., [dvips], but that may be inappropriate for some LaTeX 
%  installations. 
\usepackage[]{graphicx}
\usepackage{epstopdf}

\title{Using DARC in a Multi-Objet AO Bench and in a Dome Seeing Instrument}

%>>>> The author is responsible for formatting the 
%  author list and their institutions.  Use  \skiplinehalf 
%  to separate author list from addresses and between each address.
%  The correspondence between each author and his/her address
%  can be indicated with a superscript in italics, 
%  which is easily obtained with \supit{}.

\author{Norman S\'aez$\supit{a}^{*}$, Alastair Basden\supit{b}, Dani Guzm\'an\supit{a}, Nicol\'as Dubost\supit{a}
\skiplinehalf
\supit{a}Pontificia Universidad Cat\'olica de Chile, Centre for Astro-Engineering, Av.Vicu\~na Mackenna 4860, Santiago, Chile\\
\supit{b}University of Durham, Department of Physics, Centre for Advanced Instrumentation, South Road, Durham DH1 3LE, UK\\ $*$nfsaez@uc.cl
%\supit{b}Affiliation2, Address, City, Country
}

%>>>> Further information about the authors, other than their 
%  institution and addresses, should be included as a footnote, 
%  which is facilitated by the \authorinfo{} command.

\authorinfo{Further author information: Norman S\'aez: E-mail: nfsaez@uc.cl}
%%>>>> when using amstex, you need to use @@ instead of @
 

%%%%%%%%%%%%%%%%%%%%%%%%%%%%%%%%%%%%%%%%%%%%%%%%%%%%%%%%%%%%% 
%>>>> uncomment following for page numbers
% \pagestyle{plain}    
%>>>> uncomment following to start page numbering at 301 
%\setcounter{page}{301} 
 
  \begin{document} 
  \maketitle 

%%%%%%%%%%%%%%%%%%%%%%%%%%%%%%%%%%%%%%%%%%%%%%%%%%%%%%%%%%%%% 
\begin{abstract}
The Durham adaptive Optics Real Time Controller (DARC)\cite{basden2010durham} is a real-time system
for astronomical adaptive optics systems originally developed at Durham
University and in use for the CANARY instrument. One of its main strengths is
to be a generic and high performance real-time controller running on a
off-the-shelf Linux computer. DARC is an open-source project. We are using DARC
for two different implementations: BEAGLE, a Multi-Object AO (MOAO) bench
system to experiment with novel tomographic reconstructors and LOTUCE2, an
in-dome turbulence instrument. We present the software architecture for each
application, current benchmarks and lessons learned for current and future DARC
developers.
\end{abstract}

%>>>> Include a list of keywords after the abstract 

\keywords{DARC, Adaptive Optics,  Multi-Object Adaptive Optics, Charge Coupled Device}

%%%%%%%%%%%%%%%%%%%%%%%%%%%%%%%%%%%%%%%%%%%%%%%%%%%%%%%%%%%%%
\section{INTRODUCTION}
\label{sec:intro}  % \label{} allows reference to this section
The Durham adaptive Optics Real Time Controller (DARC) is a real-time system
for astronomical adaptive optics systems originally developed at Durham
University and in use for the CANARY instrument. One of its main strengths is
to be a generic and high performance real-time controller running on a
off-the-shelf Linux computer. It was developed in a modular way making it
flexible enough to simple or sophisticated AO instruments. Very importantly,
DARC is an open-source project. Taking advantage of it modularization degree,
it is possible to implement different AO algorithms as well as different pieces
of hardware, such as cameras for wavefront sensors (WFS) and deformable
mirrors. As it is a standard Linux application, it is possible to take
advantage of external components which can be interfaced in Linux, such as
Firewire or GigE cameras. For new components, it is always possible to build a
Linux driver and Application Program Interface that can control the device.
DARC uses current multi-core technologies efficiently, performing at speeds
that were only possible to achieve with complex architectures in the past\cite{basden2012durham}. 
We are using DARC for two different implementations: BEAGLE and LOTUCE2. BEAGLE
(presented elsewhere at this conference) is a Multi-Object AO (MOAO) bench
system to experiment with novel tomographic reconstructors, such as Artificial
Neural Networks. For BEAGLE we have developed a new module within DARC, which
runs the bench in ``multiple WFS'' mode, by controlling a constellation of light
sources, three phase plates mechanisms and only one WFS camera. DARC runs this
system in conjunction with a Single Board Computer which is in charge of some
of the hardware components. LOTUCE2 (presented elsewhere at this conference) is
an in-dome turbulence instrument, which measures the angle-of-arrival of
multiple lasers to individual high-speed cameras. DARC has been used to run
four cameras, which have in-sync hardware trigger. DARC processes the pixel
streams, obtaining the angle-of-arrival from each camera, saving data to disk
and performing various analysis in real-time. We present the software
architecture for each application, current benchmarks and lessons learned for
current and future DARC developers.

\subsection{BEAGLE}
BEAGLE is a Multi-Object AO (MOAO) bench that can emulate the optical
characteristics of the William Hershel Telescope, it has two motorized phase
plates which allow change the position of the plate scale. DARC, a real-time
system for astronomical adaptive optics systems, is beeing used to control
acquisition images. Extra development was necessary integrate specific hardware
devices component which represents phase plate movement and guide stars. The
main features which DARC provides and BEAGLE uses are the centroid calculation
in real-time\cite{basden2012wavefront}, and camera control for Firewire
protocol. It was pending create such a way to controls new peripherals involved
to make BEAGLE works. Toward that objective, a single board computer was used.
The selected single board computer was a BeagleBone Back (BBB). The challenge
of integrating DARC with new the single board computer was relative simple, and
relays on a local area network (TCP/IP connectivity) and scripting language
glue. 


\subsection{Lotuce2}
Lotuce (LOcal Turbulence Experiement)\cite{ziad1a2013lotuce} is an experimental
concept to mesure and characterized the optical-turbulence inside a telescope
enclosureor has been developed\cite{berdja1afirst}. LOTUCE2 is an upgraded
prototype whose main aim is to measure optical turbulence characteristics more
precisely by minimising cross-contamination of signals. This characterisation
is both quantitative (optical turbulence strength) and qualitative (assessing
the optical turbulence statistical model). 
% in order to measure and characterize the so-called dome-seeing.


\section{Software Architecture}\label{sec:SWA} 
\subsection{BEAGLE}
BEAGLE is being controlled by The Durham adaptive Optics Real
TimeController(DARC). DARC is a real-time system for astronomical adaptive
optics systems originally developed at Durham University and in use for the
CANARYinstrument. At this moment, we are using DARC to control acquisition
images.Extra development was necessary integrate specific hardware devices
component.BEAGLE has to integrate phase screens movements and laser star
guides.The main features which DARC provides and BEAGLE uses are the centroid
calculation in real time, and the camera control for Firewire protocol. It was
pending create such a way to controls new peripherals involved to make BEAGLE
works.Toward that objective, a single board computer was used. The selected
single board computer was a BeagleBone Back (BBB). It was chosen due low cost
HW, Linuxembedded, network connectivity, a huge amount of GPIO. The challenge of
integrating DARC with new the single board computer was relative simple, and
relays on a local network (TCP/IP connectivity). As  DARC, omniORB was installed
in the BBB; this allows develop a CORBA server, which controls most of the
peripherals. Real time software is not required in BBB, due the way of the
acquisition methods.  For the BBB CORBA server, Python language was used.


  BBB Server communicates with GPIO of BBB that are connected to motors and
  leds (phase screen and stars respectively). The performance on motors  wasn't
  good enough using Python libraries. To improves performance issue, a  C-Python
  extension was developed. This increase phase screen movement speed.  The
  extension C-Python extension allows to use Python server importing native  it,
  controlling GPIO directly.
    Using BBB CORBA server and taking advantage of CORBA server provided in
    DARC,  it was relative easy make both pieces of software work together. DARC
    was  installed in a powerful machine called RTCAOLAB, and uses the exposed
    methods  of BBB, controlling all HW in a single machine, even tough that
    each  peripheral is connected to different machines.  Figure [XXX] shows
    architecture used to controls peripherals in BBB  Server, and how RCTAOLAB
    communicates with BBB.
      
    The method developed exposed briefly here, has a positive impact in future
    improvements. The server, calibration scripts, acquisition scripts and GUIs
    are disengaged enough to change the BBB with another single board computer
    with no significant impact, just adapting a few specific low level software
    layers for the new board.
\subsection{Lotuce2}
Lotuce2 is being controlled by The Durham adaptive Optics Real Time
Controller(DARC). DARC is a real-time system for astronomical adaptive optics
systems originally developed at Durham University and in use for the CANARY
instrument.

Initially, Lotuce used a single camera, but Lotuce2 uses four
cameras. The core functionality being used by Lotuce2 and camera acquisition
and slope calculation, all provided by DARC.  

For Lotuce2, DARC was installed
in two different machines. It was installed in a Mac Book pro Late2011, and a
desktop machine.  Different operative system were used, but the same software
were compiled over them. Desktop machine were used a Fedora 14 Linux kernel
2.6.35.13-91 32 bits architecture. Mac Book pro instead has installed a Linux
Ubuntu 13.10 kernel 3.x.x.x 64 bits architecture.  Besides this difference, the
behaviour was exactly the same. Notice that each kernel is not patched as real
time, but is was good enough for Lotuce2 purposes.

DARC is highly flexible with each camera protocol, it can handle USB,
iee1394 cameras (fire-wire cameras) as well as Gig protocols. Cameras used
are two Gigabit Ethernet protocol, an a Cisco 3500 was configured according
to plug all possible cameras. This switch was transparent for DARC for
either installation.  

Camera synchronization were a defiance to face. The approach taken was an
external trigger shared among the cameras. The device used to trigger the
camera acquisition is a single board computer, integrated and configurable GPIO.
Beagle Bone Black (BBB) was the selected one for these purposes. 

BBB has an ARM Cortex-A8 processor, with a Linux inside. BBB uses an embedded
Ubuntu 13.10 Linux . 100 [Khz] is the  maximum frequency reached turning on off
a GPIO. To improve this situation, a Programmable Real-time Unit(PRU) was used.
This unit performs embedded tasks that require real-time constraints. PRU is a
subsystem of the processor inside BBB. It is an independent CPU with its own
memory and instruction set. It can run its own program, completely independent
of the Linux kernel on the main CPU. Using PRU approach BBB could reach a signal
frequency of 200[MHz], accomplishing  with Lotuce22 requirements.  

DARC can get both camera data in a single stream, one camera stream next to the
other. It is mandatory that each camera protocol has its own driver
implementation make it compatible with DARC. Aravis is a library for video
acquisition which are being used to make the driver work in DARC. Preliminary
shows promising results, frames per seconds for our purposes should be up to
125. Using both cameras at same time (but not at maximum resolution) we are
getting results up to 200fps.  The trigger to acquire an image with cameras is
working fine but is still under testing. See figure [XXX] that shows the
current design. Slight modifications are not discard at this time.  

The main program should be running were DARC is installed (either Mac Book pro
or Desktop) and the signal trigger it will be send using TCP/IP protocol,
arriving at BBB, triggering the acquisition. Some automatic routines are being
implemented, like automatic sub aperture location, ideal image size and some
graphics plots to help to controls the instrument operations. Image [ZZZ] shows
prototype GUI examples.

%%-------------
%   \begin{figure}[!ht]
%   \begin{center}
%   \begin{tabular}{c}
%   \includegraphics[height=1.8cm]{../img/reg-tests-workflow-2.png}
%   \end{tabular}
%   \end{center}
%   \caption[workflow] 
%%>>>> use \label inside caption to get Fig. number with \ref{}
%   { \label{fig:reg-tests-workflow} 
%Current structure of the Regression Tests Procedure}
%   \end{figure} 
%%-------------


%\newpage
%%%%%%%%%%%%%%%%%%%%%%%%%%%%%%%%%%%%%%%%%%%%%%%%%%%%%%%%%%%%%
\section{Benchmarks} \label{sec:benchmarks}
\subsection{BEAGLE}
\subsection{Lotuce2}

%%-------------
%   \begin{figure}[!ht]
%   \begin{center}
%   \begin{tabular}{c}
%   \includegraphics[height=10cm]{../img/casa-reduction.png}
%   \end{tabular}
%   \end{center}
%   \caption[casa] 
%%>>>> use \label inside caption to get Fig. number with \ref{}
%   { \label{fig:casa-reduction} 
%Example of fringe pattern validation by using CASA off-line tool.}
%   \end{figure} 
%%-------------

%%%%%%%%%%%%%%%%%%%%%%%%%%%%%%%%%%%%%%%%%%%%%%%%%%%%%%%%%%%%%
\section{Lesson Learned} \label{sec:LL}
\subsection{BEAGLE}
\subsection{Lotuce2}

%%-------------
%   \begin{figure}[!ht]
%   \begin{center}
%   \begin{tabular}{c}
%   \includegraphics[height=9cm]{../img/UMLmodel.png}
%   \end{tabular}
%   \end{center}
%   \caption[uml] 
%%>>>> use \label inside caption to get Fig. number with \ref{}
%   { \label{fig:UMLmodel} 
%UML diagram of the Automatic Regression Testing framework.}
%   \end{figure} 
%%-------------


%\newpage
\section{Conclusions}

%\appendix    %>>>> this command starts appendixes
%%%%%%%%%%%%%%%%%%%%%%%%%%%%%%%%%%%%%%%%%%%%%%%%%%%%
%%%%%%%%%%%%%%%%%%%%%%%%%%%%%%%%%%%%%%%%%%%%%%%%%%%%%%%%%%%%%
%\acknowledgments     %>>>> equivalent to \section*{ACKNOWLEDGMENTS}       
 
%[...]

%%%%%%%%%%%%%%%%%%%%%%%%%%%%%%%%%%%%%%%%%%%%%%%%%%%%%%%%%%%%%
%%%%% References %%%%%

\newpage
\acknowledgments{Hi}
\bibliography{report}   %>>>> bibliography data in report.bib
\bibliographystyle{spiebib}   %>>>> makes bibtex use spiebib.bst

\end{document} 
